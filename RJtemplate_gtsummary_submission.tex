% !TeX root = RJwrapper.tex
\title{Reproducible Summary Tables with the e \pkg{gtsummary} Package}
\author{by Daniel D. Sjoberg, Karissa Whiting, Michael Curry}

\maketitle

\abstract{
The \pkg{gtsummary} package provides an elegant and flexible way to create publication-ready summary tables in R. The motivation behind the package stems from the work of statisticians, data scientists, and analysts, where every day, they summarize datasets and regression models in R, share results with collaborators, and eventually include them in published manuscripts or share with stakeholders. The \pkg{gtsummary} streamlines this process creating summaries of data sets, regression models, survey data, and survival data with simple code resulting in beautiful tables suitable for publication--often with a single line of code. The package follows a tidy framework, yielding an interface that is easy to learn that is also highly customizable.
}

\section{Introduction}
A common need in research and business analytics is summarizing data and statistical models.
Table summaries identify patterns in the data that help analysts discover insights and drive action.
While there are options to create summary tables in R, few, if any packages, have the ability to make tables that are easy to: export, interpret, reproduce, and create. 

The \CRANpkg{gtsummary} \citep{gtsummary} package provides an elegant and flexible way to create publication-ready analytical and summary tables in R.
This package works with existing packages and frameworks to close the loop between reproducible RMarkdown R code and polished, organized statistical reports.
Specifically, gtsummary allows the users to fully customize the format of tables in R so users do not have to modify any tables after an RMarkdown has rendered or a table has been exported.
Getting rid of the need to modify tables after the table has been created, reduces an error-prone workflow and increases the chances that your analysis is reproducible. 

Using \pkg{gtsummary}, analysts can easily summarize data frames, present and compare descriptive statistics between groups, summarize the most common types of regression models, and report statistics in line in R markdown reports. 
Identifying building blocks of tables presented in medical literature (that also are common among all fields), \pkg{gtsummary} seeks to create these tables and their formats so there is no need for further end-user modifications.  

Additionally, \pkg{gtsummary} leverages other data cleaning and analysis packages to create a complete analysis and reporting framework.
For example, users can take advantage of existing broom tidiers to clean regression results for \texttt{tbl\_regression()} and use \pkg{gt}, \pkg{kable}, \pkg{flextable} or others to print \pkg{gtsummary} tables to various output formats (e.g. HTML, PDF, Word, RTF).
Furthermore, all \pkg{gtsummary} functions are created to work within a “tidy” framework where a data frame is the first argument and functions can be used with pipes.
Unquoted variables are used and tidyselect functions can be used to select and manipulate groups of variables within the framework.

While other R packages are available to clean and present summary statistics and regression results in statistical reports, \pkg{gtsummary} is unique in that it is a one-stop-shop for all types of statistical tables and offers diverse features to customize content and format of tables to a high degree.
For example, \pkg{finalfit} only creates regressions outputs and you have to manually specify a printing engine for the table to render.
Furthermore, \pkg{tableone} reports on general descriptive statistics, it is fundamentally lacking in its ability to produce tables for regression modeling strategies in R. 

\pkg{gtsummary} comes equipped common descriptive summaries for several data types allowing the user to easily aggregate data.
At first, \pkg{gtsummary} will identify data as continuous or discrete and apply summary methods to each variable. 
However, any default in identifying data types can be manually overridden. 
Additionally, if any descriptive statistical method is not inherently in \pkg{gtsummary}, the user can supply their own function to summarise the data. 
Along with common descriptive summaries, \pkg{gtsummary} has the most common methods to summarise statistical models, survey data, and even perform cross tables for comparing discrete variables. 
After data summarization, \pkg{gtsummary} allows users to combine tables, either side by side (with \texttt{tbl\_merge()}) , or on top of each other (with \texttt{tbl\_stack()}). 
This data merging allows users to easily synthesize and compare output from several tables and share a lot of information in a compact format.   

\section{Data Summaries}

To show use of \pkg{gtsummary} functions, we will use a simulated clinical trial data set containing baseline characteristics of 200 patients who received Drug A or Drug B as well as the outcome of tumor response to the treatment.
Each variable in the data frame has been assigned an attribute label (i.e. \texttt{attr(trial\$trt, "label") == "Chemotherapy Treatment"}) with the labelled package \citep{labelled}. 
These labels are displayed in the \pkg{gtsummary} tables by default. Using \pkg{gtsummary} on a data frame without labels will print variable names in place of variable labels with the option to update the text displayed.

% adding table describing trial dataset
\captionsetup[table]{labelformat=empty,skip=1pt}
\begin{longtable}{llll}
\caption{\label{tab:}Example data frame, \texttt{trial}}
\toprule
colname & label & class & values \\ 
\midrule
\texttt{\textquotedbl{}trt\textquotedbl{}} & Chemotherapy Treatment & character & \texttt{Drug A}, \texttt{Drug B} \\ 
\texttt{\textquotedbl{}age\textquotedbl{}} & Age & numeric & \texttt{6}, \texttt{9}, \texttt{10}, \texttt{17} , ... \\ 
\texttt{\textquotedbl{}marker\textquotedbl{}} & Marker Level (ng/mL) & numeric & \texttt{0.003}, \texttt{0.005}, \texttt{0.013}, \texttt{0.015} , ... \\ 
\texttt{\textquotedbl{}stage\textquotedbl{}} & T Stage & factor & \texttt{T1}, \texttt{T2}, \texttt{T3}, \texttt{T4} \\ 
\texttt{\textquotedbl{}grade\textquotedbl{}} & Grade & factor & \texttt{I}, \texttt{II}, \texttt{III} \\ 
\texttt{\textquotedbl{}response\textquotedbl{}} & Tumor Response & integer & \texttt{0}, \texttt{1} \\ 
\texttt{\textquotedbl{}death\textquotedbl{}} & Patient Died & integer & \texttt{0}, \texttt{1} \\ 
\texttt{\textquotedbl{}ttdeath\textquotedbl{}} & Months to Death/Censor & numeric & \texttt{3.53}, \texttt{5.33}, \texttt{6.32}, \texttt{7.27} , ... \\ 
\bottomrule
\end{longtable}



\subsection{\texorpdfstring{\texttt{tbl\_summary()}}{tbl\_summary()}}

The default output from \texttt{tbl\_summary()} is meant to be publication ready.
The \texttt{tbl\_summary()} function can take, at minimum, a data frame as the only input, and returns descriptive statistics for each column in the data frame.
This is often the first table of clinical manuscripts and describes characteristics of the cohort under study.
A simple example is shown below.
Notably, by specifying the \texttt{by=} argument, you can stratify the summary table. 
In the example below, we have split the table by the treatment a patient received. 

% code for basic tbl_summary()
\begin{example}
trial %>%
  select(age, grade, response, trt) %>%
  tbl_summary(by = trt)
\end{example}
\begin{figure}[h!]
  \includegraphics[scale=0.28]{summary_basic.png}
  \centering
\end{figure}

The function is highly customizable, and it is initiated with sensible default settings.
Specifically, \texttt{tbl\_summary} detects variable types of input data and calculates descriptive statistics accordingly.
For example, variables coded as \texttt{0/1}, \texttt{TRUE/FALSE}, and \texttt{Yes/No} are presented dichotomously.
Additionally, \texttt{NA} values are recognized as missing and listed as unknown, and if a data set is labelled, the label attributes are automatically utilized. 

Default settings may be customized using the \texttt{tbl\_summary()} function arguments.

% code for tbl_summary() with arguments
\captionsetup[table]{labelformat=empty,skip=1pt}
\begin{longtable}{ll}
\caption{\label{tab:}\texttt{tbl\_summary()} function arguments}\\
\toprule
Argument & Description \\ 
\midrule
\texttt{label} & specify the variable labels printed in table \\ 
\texttt{type} & specify the variable type (e.g. continuous, categorical, etc.) \\ 
\texttt{statistic} & change the summary statistics presented \\ 
\texttt{digits} & number of digits the summary statistics will be rounded to \\ 
\texttt{missing} & whether to display a row with the number of missing observations \\ 
\texttt{missing\_text} & text label for the missing number row \\ 
\texttt{sort} & change the sorting of categorical levels by frequency \\ 
\texttt{percent} & print column, row, or cell percentages \\ 
\bottomrule
\end{longtable}



In the example below, continuous variables are cast to \texttt{"continuous2"}, meaning the continuous summary statistics will appear on two or more rows in the table.
The \texttt{"age"} variable's label is updated to \texttt{"Patient Age"}.
Default summary statistics for both continuous and categorical variables are updated using the \texttt{statistic=} argument. ADD SOMETHING ABOUT GLUE SYNTAX!
The \texttt{digits=} argument is used to increase the number of decimal places the statistics are rounded, and the missing row is omitted with \texttt{missing = "no"}.

\begin{example}
trial %>%
  select(age, grade, response, trt) %>%
  tbl_summary(
    by = trt,
    type = all_continuous() ~ "continuous2",
    label = age ~ "Patient Age",
    statistic = list(all_continuous() ~ c("{N_nonmiss}", 
                                          "{mean} ({sd})", 
                                          "{median} ({p25}, {p75})", 
                                          "{min}, {max}"),
                     all_categorical() ~ "{n} / {N} ({p}%)"),
    digits = all_categorical() ~ c(0, 0, 1),
    missing = "no"
  )
\end{example}
\begin{figure}[h!]
  \includegraphics[scale=0.28]{summary_plus.png}
  \centering
\end{figure}

\textbf{A note about notation:}
Throughout the \pkg{gtsummary} package, you'll find function arguments that accept a list of formulas (or a single formula) as the input.
In the example above, the label for the age variable was updated using \texttt{label = age $\sim$ "Patient Age"}---equivalently, \texttt{label = list(age $\sim$ "Patient Age")}.
To select groups of variables, utilize the select helpers from the tidyselect \citep{tidyselect} and \pkg{gtsummary}.
Above, \texttt{all\_continuous()} was utilized to change the summary statistics for all continuous variables. 
Similarly, users may utilize \texttt{all\_categorical()} (from \pkg{gtsummary}), or any of the tidyselect helpers used throughout the tidyverse \citep{tidyverse} package, such as \texttt{starts\_with()}, \texttt{contains()}, etc.

The \pkg{gtsummary} package has several functions to add information or statistics to \texttt{tbl\_summary()} tables.

% code for tbl_summary() with arguments
\input{tbl_tbl_summary_family.tex}

In the example below the number of non-missing observations is reported for each variable, as well as a p-value comparing the values between the treatment.
The \texttt{add\_p(pvalue\_fun=)} argument accept both a proper function as well the formula shortcut notation used through the tidyverse packages.

\begin{example}
trial %>%
  select(age, grade, response, trt) %>%
  tbl_summary(by = trt, missing = "no") %>%
  add_n() %>%
  add_p(test = all_continuous() ~ "t.test",
        pvalue_fun = ~style_pvalue(., digits = 2))

\end{example}
\begin{figure}[h!]
  \includegraphics[scale=0.28]{summary_plus_plus.png}
  \centering
\end{figure}

\subsection{\texorpdfstring{\texttt{tbl\_svysummary()}}{tbl\_svysummary()}}

The \texttt{tbl\_svysummary()} function is very similar to \texttt{tbl\_summary()} except a survey object is supplied rather than a data frame.

\begin{example}
# create weighted survey object, keep only two counties
data(api, package = "survey")
svy_api <- 
  survey::svydesign(id = ~dnum, weights = ~pw, data = apiclus1, fpc = ~fpc) %>%
  subset(cname %in% c("Alameda", "Los Angeles"))

# summarize data
svy_api %>%
  tbl_svysummary(by = cname, 
                 include = c(growth, target, cname),
                 label = list(growth ~ "Growth",
                              target ~ "Target")) %>%
  add_p()
\end{example}
\begin{figure}[h!]
  \includegraphics[scale=0.28]{svysummary.png}
  \centering
\end{figure}

\subsection{\texorpdfstring{\texttt{tbl\_cross()}}{tbl\_cross()}}

The \texttt{tbl\_cross()} function continues using similar syntax as the previous functions, resulting in a cross tabulation table ready for publication. 

\begin{example}
trial %>%
  tbl_cross(row = stage, col = trt, percent = "cell") %>%
  add_p(source_note = TRUE)
\end{example}
\begin{figure}[h!]
  \includegraphics[scale=0.28]{cross.png}
  \centering
\end{figure}

\subsection{\texorpdfstring{\texttt{tbl\_survfit()}}{tbl\_survfit()}}

The \texttt{tbl\_survfit()} parses and tabulates n-year survival and survival percentile estimates from \texttt{survival::survfit()} objects.

\begin{example}
library(survival)

list(survfit(Surv(ttdeath, death) ~ trt, trial),
     survfit(Surv(ttdeath, death) ~ grade, trial)) %>%
  tbl_survfit(times = c(12, 24),
              label_header = "**{time} Month**") %>%
  add_p()
\end{example}

% TODO: I do not understand why this isn't being placed below the code! https://www.overleaf.com/learn/latex/Positioning_of_Figures
\begin{figure}[h!]
  \includegraphics[scale=0.28]{survfit.png}
  \centering
\end{figure}

\subsection{Customization}

The \pkg{gtsummary} package includes functions specifically made to modify and format the summary tables.
These functions work with any table constructed with \pkg{gtsummary}.
Most common uses are changing the column headers and footnotes.

% adding table describing fns for customization
\captionsetup[table]{labelformat=empty,skip=1pt}
\begin{longtable}{ll}
\caption{\label{tab:} Functions to style and modify gtsummary tables}\\
\toprule
Function & Description \\ 
\midrule
\texttt{modify\_header()} & update column headers \\ 
\texttt{modify\_footnote()} & update column footnote \\ 
\texttt{modify\_spanning\_header()} & update spanning headers \\ 
\texttt{modify\_caption()} & update table caption/title \\ 
\texttt{bold\_labels()} & bold variable labels \\ 
\texttt{bold\_levels()} & bold variable levels \\ 
\texttt{italicize\_labels()} & italicize variable labels \\ 
\texttt{italicize\_levels()} & italicize variable levels \\ 
\texttt{bold\_p()} & bold significant p-values \\ 
\bottomrule
\end{longtable}



The \pkg{gtsummary} package utilizes the gt package \citep{gt} to print the summary tables.
The gt package exports approximately one hundred functions to customize and style tables.
When you need to add additional details or styling not available within \pkg{gtsummary}, use the \texttt{as\_gt()} to convert the \pkg{gtsummary} object to gt and continue customization.

\begin{example}
trial %>%
  select(age, grade, trt) %>%
  tbl_summary(by = trt, missing = "no") %>%  
  add_overall() %>%    
  add_p() %>%      
  add_stat_label() %>%     
  modify_header(label ~ "**Variable**") %>%   
  modify_spanning_header(c(stat_1, stat_2) ~ "**Treatment Received**") %>%  
  as_gt() %>% 
  gt::tab_header(   
    title = gt::md("**Table 1. Patient Characteristics**"),
    subtitle = gt::md("_Highly Confidential_")
  ) %>%
  gt::tab_source_note("Data updated June 26, 2015")   
\end{example}
\begin{figure}[h!]
  \includegraphics[scale=0.28]{custom.png}
  \centering
\end{figure}

\section{Model Summaries}

Regression modelling is one of the most common tools of medical researchers.
The \pkg{gtsummary} has two functions to assist researchers and analysts to prepare tabular summaries of regression models: \texttt{tbl\_regression()} and \texttt{tbl\_uvregression()}.

\subsection{\texorpdfstring{\texttt{tbl\_regression()}}{tbl\_regression()}}

The \texttt{tbl\_regression()} function takes a regression model object in R and returns a formatted table of regression model results. 
Like \texttt{tbl\_summary()}, \texttt{tbl\_regression()} creates highly customizable analytic tables with sensible defaults.
Common regression models, such as logistic regression and Cox proportional hazards regression, are automatically identified and the tables headers are pre-filled with appropriate column headers (i.e. Odds Ratio and Hazard Ratio).

In the example below, the logistic regression model is summarized with \texttt{tbl\_regression()}.
Note that reference row for Grade I has been added, and the variable labels have been carried through into the table.
Using \texttt{exponentiate = TRUE} exponentiates the regression coefficients yielding the odds ratios.
The helper function \texttt{add\_global\_p()} was used to replace the p-values for each term with the global p-value for grade.

\begin{example}
glm(response ~ age + grade, trial, family = binomial) %>%
  tbl_regression(exponentiate = TRUE) %>%
  add_global_p()
\end{example}

\begin{figure}[h!]
  \includegraphics[scale=0.28]{regression.png}
  \centering
\end{figure}

The \texttt{tbl\_regression()} function leverages the huge effort behind the broom package's \texttt{tidy()} function \citep{broom} to perform the initial formatting of the regression object.
Because \texttt{tbl\_regression()} utilizes \texttt{broom::tidy()}, there are many model types that are supported out of the box, such as \texttt{lm()}, \texttt{glm()}, \texttt{lme4::lmer()}, \texttt{lme4::glmer()}, \texttt{geepack::geeglm()}, \texttt{survival::coxph()}, \texttt{survival::survreg()}, \texttt{survival::clogit()}, \texttt{nnet::multinom()}, \texttt{rstanarm::stan\_glm()}, models built with the mice package[add ref]and more. The \pkg{gtsummary} package also contains a handful of custom tidiers to report standardized coefficients and bootstrapped confidence intervals.

\subsection{\texorpdfstring{\texttt{tbl\_uvregression()}}{tbl\_uvregression()}}

The \texttt{tbl\_uvregression()} function is a wrapper for \texttt{tbl\_regression()} that is useful when you need a series of univariate regression models.
The user passes a data frame to \texttt{tbl\_uvregression()}, indicates what the outcome is, what regression model to run, and the function will return a beautifully formatted table of stacked univarate regression models.

\begin{example}
trial %>%
  select(response, age, grade) %>%
  tbl_uvregression(
    y = response, 
    method = glm,
    method.args = list(family = binomial),
    exponentiate = TRUE,
    pvalue_fun = ~style_pvalue(., digits = 2)
  ) %>%
  add_nevent() %>%
  add_global_p()
\end{example}

\begin{figure}[h!]
  \includegraphics[scale=0.28]{uvregression.png}
  \centering
\end{figure}

\section{Inline Reporting}

Reproducible reports are an important part of good practices.
We often need to report the results from a table in the text of an R markdown report.
Inline reporting has been made simple with \texttt{inline\_text()}.
The function reports statistics from \pkg{gtsummary} tables inline in an R markdown document.

Imagine you need to report the results for age from the univariate table above.
Typically, the odds ratio, confidence interval, and p-value would be hard-coded into a report, which can lead to reproducibility issues if the data is updated and the hard-coded statistics are not amended.
A simple call to the \texttt{inline\_text()} function will dynamically add the model results to a an Rmarkdown report.

\begin{quote}
The odds ratio for age was \texttt{\textasciigrave{}r\ inline\_text(uvreg,\ variable\ =\ age)\textasciigrave{}}.
\end{quote}

Here's how the line will appear in your report.

\begin{quote}
The odds ratio for age was 1.02 (95\% CI 1.00, 1.04; p=0.091).
\end{quote}

The default pattern to display for a regression table is \texttt{"\{estimate\} (\{conf.level*100\}\% CI \{conf.low\}, \{conf.high\}; \{p.value\})"} (again using glue syntax), and can be modified with the \texttt{inline\_text(pattern=)} argument. 

\section{Merging and Stacking}

\section{Themes}

The default styling (e.g. statistics displayed in \texttt{tbl\_summary()}, how p-values are rounded, decimal separator, and more) follow the reporting guidelines from European Urology, The Journal of Urology, Urology, and the British Journal of Urology International \citep{assel2019guidelines}.
However, you'll likely submit to another journal, or your personal preferences different from the defaults.
The \pkg{gtsummary} package is unique from other table building packages in the ability to set fine-grained customization defaults with themes. 
Themes were created to make these customizations easy to navigate and reuse across documents or projects. 
Several ready built themes are available in the package.
With themes, users can control default settings for existing functions (e.g. always present means instead of medians in \texttt{tbl\_summary()}), as well as other changes that are not modifiable with function arguments.

For example, using the theme for New England Journal of Medicine, large p-values are rounded to two decimal places and confidence intervals are shown as \texttt{"lb to ub"} instead of \texttt{"lb, ub"}. A theme was used to construct the tables shown in this manuscript to change the font to match the \emph{R Journal} font.

\begin{example}
theme_gtsummary_journal("nejm")

glm(response ~ age + grade, trial, family = binomial) %>%
  tbl_regression(exponentiate = TRUE)
\end{example}

\begin{figure}[h!]
  \includegraphics[scale=0.28]{nejm.png}
  \centering
\end{figure}

The themes are an evolving feature and we welcome additions of new journals and other useful themes.
A full glossary of customizable theme elements is available in the package documentation (\url{http://www.danieldsjoberg.com/gtsummary/articles/themes.html}).

\section{Print Engines}

The \pkg{gtsummary} package 
SHOULD WE JUST REMOVE THIS SECTION?!

\begin{figure}[h!]
  \includegraphics[scale=0.28]{print_engines.png}
  \centering
\end{figure}

\section{Summary}

\bibliography{RJreferences}

\address{Daniel D. Sjoberg\\
  Memorial Sloan Kettering Cancer Center\\
  1275 York Ave., New York, New York 10022\\
  USA\\
  ORCiD: 0000-0003-0862-2018\\
  \email{sjobergd@mskcc.org}}

\address{Karissa Whiting\\
  Memorial Sloan Kettering Cancer Center\\
  1275 York Ave., New York, New York 10022\\
  USA\\
  ORCiD: 0000-0002-4683-1868\\
  \email{whitingk@mskcc.org}}

\address{Michael Curry\\
  Memorial Sloan Kettering Cancer Center\\
  1275 York Ave., New York, New York 10022\\
  USA\\
  ORCiD: 0000-0002-0261-4044\\
  \email{currym1@mskcc.org}}
